

%\runauthor{Xin, Qin and Sun}

\begin{frontmatter}

\title{A Neural Network-Evolutionary Computational Framework for Remaining Useful Life Estimation of Mechanical Systems}

\author{David Laredo$^{1}$, Zhaoyin Chen$^{1}$, Oliver Sch\"utze$^{2}$ and Jian-Qiao Sun$^{1}$}
\address{
$^{1}$Department of Mechanical Engineering\\
School of Engineering, University of California\\
Merced, CA 95343, USA\\
$^{2}$Department of Computer Science, CINVESTAV\\ 
Mexico City, Mexico\\
Corresponding author. Email: davidlaredo1@gmail.com}

\begin{abstract}
This paper presents a framework for estimating the remaining useful life (RUL) of mechanical systems. The framework consists of a multi-layer perceptron and an evolutionary algorithm for optimizing the data-related parameters. The framework makes use of a strided time window to estimate the RUL for mechanical components. Tuning the data-related parameters can become a very time consuming task. The framework presented here automatically reshapes the data such that the efficiency of the model is increased. Furthermore, the complexity of the model is kept low, e.g. neural networks with few hidden layers and few neurons at each layer. Having simple models has several advantages like short training times and the capacity of being in environments with limited computational resources such as embedded systems. The proposed method is evaluated on the publicly available C-MAPSS dataset \cite{Saxena2008a}, its accuracy is compared against other state-of-the art methods for the same dataset. 


\end{abstract}


\begin{keyword}
artificial neural networks\sep
moving time window\sep
RUL estimation\sep
prognostics\sep
evolutionary algorithms
\end{keyword}

\end{frontmatter}
